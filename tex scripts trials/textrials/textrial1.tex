\documentclass[11pt, a4paper]{article}

\usepackage[a4paper, margin=1in]{geometry}
\usepackage{graphicx}
\usepackage{amsmath}
\usepackage{amssymb}
\usepackage{xeCJK}
\usepackage{enumitem}
\usepackage{booktabs}
\usepackage{parskip}
\usepackage{fancyhdr}
\usepackage{xcolor}
\usepackage{array}
\usepackage{makecell}
\usepackage{multirow}
\usepackage{mathrsfs}
\usepackage{bigstrut}

% Set a Chinese font available on your system (adjust if needed)
\setCJKmainfont{SimSun}

\definecolor{questioncolor}{RGB}{0,102,204}
\definecolor{optioncolor}{RGB}{153,0,0}

\setlist[enumerate]{label=\arabic*., leftmargin=*, itemsep=1.2em}
\setlist[itemize]{leftmargin=*}

\pagestyle{fancy}
\fancyhf{}
\rhead{title} % replace with actual title if needed
\rfoot{第~\thepage~页}

\newcommand{\questiontitle}[1]{%
    \vspace{0.5em}\noindent\textcolor{questioncolor}{\textbf{#1}}\par%
    \noindent\makebox[\linewidth]{\color{questioncolor}\rule{\paperwidth}{0.4pt}}\par%
    \vspace{0.5em}%
}

\newcommand{\options}[4]{%
    \renewcommand{\arraystretch}{1.6}%
    \setlength{\tabcolsep}{12pt}%
    \begin{tabular}{@{}p{0.48\linewidth}@{}p{0.48\linewidth}@{}}%
        \textcolor{optioncolor}{A.}~#1 & \textcolor{optioncolor}{B.}~#2 \\[4pt]
        \textcolor{optioncolor}{C.}~#3 & \textcolor{optioncolor}{D.}~#4
    \end{tabular}%
}

\newcommand{\multichoice}[1]{%
    \begin{itemize} #1 \end{itemize}
}

\newcommand{\fillblank}[1][2cm]{\underline{\hspace{#1}}}

\renewcommand{\arraystretch}{1.4}
\setlength{\extrarowheight}{2pt}
\allowdisplaybreaks[4]

\begin{document}

\begin{center}
    \Large\textbf{title}\\
    \vspace{0.5em}
    \large\textbf{subject}
\end{center}
\begin{document}
# 2022年普通高等学校招生全国统一考试

# 数学

\section*{一、 选择题:本题共8小题,每小题5分,共40分。在每小题给出的四个选项中,只有一项是符合题目要求的。}

\subsection*{1.若集合$M=\{x\mid\sqrt{x}<4\}$ $N=\{x\mid3x\geqslant1\}$ ,则$M\cap N=$}

\begin{enumerate}
\item 
\end{enumerate}$\{x\mid0\leqslant x<2\}$ 

\begin{enumerate}
\item 
\end{enumerate}$\{x\mid\frac{1}{3}\leqslant x<2\}$ 

\begin{enumerate}
\item 
\end{enumerate}$\{x\mid3\leqslant x<16\}$ 

\begin{enumerate}
\item 
\end{enumerate}$\{x\mid\frac{1}{3}\leq x<16\}$ 

\subsection*{2.若$i(1-z)=1$ ,则$z+\overline{z}=$}

\begin{enumerate}
\item 
\end{enumerate}-2\begin{enumerate}
\item 
\end{enumerate}-1$\begin{enumerate}
\item 
\end{enumerate}1\begin{enumerate}
\item 
\end{enumerate}2

\subsection*{3.在△ABC中,点D在边$AB 上$ $BD=2DA$ .记$\overrightarrow{CA}=m$ ,$\overline{CD}=n$ ,则$\overrightarrow{CB}=$}

\begin{enumerate}
\item 
\end{enumerate}$3m-2m$ \begin{enumerate}
\item 
\end{enumerate}$-2m+3n$ \begin{enumerate}
\item 
\end{enumerate}$3m+2n$ \begin{enumerate}
\item 
\end{enumerate}$2m+3n$ 

4南水北调工程缓解了北方一些地区水资源短缺问题,其中一部分水蓄入某水库.已知该水库水位为海拔148.5m时,相应水面的面积为140.0km;水位为海拔157.5m 时,相应水面的面积为$180.0km^2$ .将该水库在这两个水位间的形状看作一个棱台,则该水库水位从海拔148.5m上升到157.5m时,增加的水量约为($\sqrt{7}\approx2.65$ 

\begin{enumerate}
\item 
\end{enumerate}$1.0\times10^{9}m^{3}$ \begin{enumerate}
\item 
\end{enumerate}$1.2\times10^{9}m^{3}$ \begin{enumerate}
\item 
\end{enumerate}$1.4\times10^{9}m^{3}$ \begin{enumerate}
\item 
\end{enumerate}$1.6\times10^{9}m^{3}$ 

\subsection*{5.从2至8的7个整数中随机取2个不同的数,则这2个数互质的概率为}

\begin{enumerate}
\item 
\end{enumerate}$\frac{1}{6}$ \begin{enumerate}
\item 
\end{enumerate}$\frac{1}{3}$ \begin{enumerate}
\item 
\end{enumerate}$\frac{1}{2}$ \begin{enumerate}
\item 
\end{enumerate}$\frac{2}{3}$ 

---


的图像关于点$(\frac{3\pi}{2},2)$ 中心对称,则$f(\frac{\pi}{2})=$ 

\begin{enumerate}
\item 
\end{enumerate}1\begin{enumerate}
\item 
\end{enumerate}$\frac{3}{2}$ \begin{enumerate}
\item 
\end{enumerate}$\frac{5}{2}$ \begin{enumerate}
\item 
\end{enumerate}3

\subsection*{7.设$a=0.1e^{0.1}$ $b=\frac{1}{9},\quad c=-\ln0.9$ ,则}

\begin{enumerate}
\item 
\end{enumerate}$a<b<c$ \begin{enumerate}
\item 
\end{enumerate}$c<b<a$ \begin{enumerate}
\item 
\end{enumerate}$c<a<b$ \begin{enumerate}
\item 
\end{enumerate}$a<c<b$ 

\subsection*{8.已知正四棱锥的侧棱长为1,其各顶点都在同一球面上,若该球的体积为36π,且$3\leq1\leq3\sqrt{3}$ ,则该正四棱锥体积的取值范围是}



\begin{enumerate}
\item 
\end{enumerate}$[18,\frac{81}{4}]$ \begin{enumerate}
\item 
\end{enumerate}$[\frac{27}{4},\frac{81}{4}]$ \begin{enumerate}
\item 
\end{enumerate}$[\frac{27}{4},\frac{64}{3}]$ \begin{enumerate}
\item 
\end{enumerate} [18,27]

\section*{二、选择题:本题共4小题,每小题5分,共20分。在每小题给出的选项中,有多项 符合题目要求。全部选对的得5分,部分选对的得2分,有选错的得0分。}

\subsection*{9.已知正方体$ABCD-A_{1}B_{1}C_{1}D_{1}$ ,则}

\begin{enumerate}
\item 
\end{enumerate}直线$BC_{1}$ 与$DA_{1}$ 所成的角为$90^{\circ}$ 

\begin{enumerate}
\item 
\end{enumerate}直线$BC_{1}$ 与$CA_{1}$ 所成的角为$90^{\circ}$ 

\begin{enumerate}
\item 
\end{enumerate}直线$BC_{1}$ 与平面$B B_{1}D_{1}D$ 所成的角为450。

\begin{enumerate}
\item 
\end{enumerate}直线$BC_{1}$ 与平面$ABCD$ 所成的角为$45^{\circ}$ 

\subsection*{10.已知函数$f(x)=x^{3}-x+1$ ,则}

\begin{enumerate}
\item 
\end{enumerate}$f(x)$ 有两个极值点\begin{enumerate}
\item 
\end{enumerate}$f(x)$ 有三个零点

\begin{enumerate}
\item 
\end{enumerate}点(0,1)是曲线$y=f(x)$ 的对称中心\begin{enumerate}
\item 
\end{enumerate}直线$y=2x$ 是曲线$y=f(x)$ 的切线

\subsection*{11.已知0为坐标原点,点A(1,1)在抛物线$\begin{enumerate}
\item 
\end{enumerate}x^{2}=2py(p>0)$ 上,过点$B(0,-1)$ 的直线交C于P,Q两点,则}



\begin{enumerate}
\item 
\end{enumerate}C的准线为$y=-1$ \begin{enumerate}
\item 
\end{enumerate}直线AB与C相切

\begin{enumerate}
\item 
\end{enumerate}$\vert O P\vert\cdot\vert O Q\vert>\vert O A\vert^{2}$ 

$$\left|B P\right|\cdot\left|B Q\right|>\left|B A\right|^{2}$$

\subsection*{12.已知函数$f(x)$ 及其导函数$f^{\prime}(x)$ 的定义域均为R,记$g(x)=f^{\prime}(x)$ .若$f(\frac{3}{2}-2x)$ $8(2+x)$ 均为偶函数,则}



\begin{enumerate}
\item 
\end{enumerate}$f\begin{enumerate}
\item 
\end{enumerate}=0$ \begin{enumerate}
\item 
\end{enumerate}$g(-\frac{1}{2})=0$ \begin{enumerate}
\item 
\end{enumerate}$f(-1)=f\begin{enumerate}
\item 
\end{enumerate}$ \begin{enumerate}
\item 
\end{enumerate}$g(-1)=g\begin{enumerate}
\item 
\end{enumerate}$ 

数学试题第2页(共4页)

---


\section*{三、 填空题:本题共4小题,每小题5分,共20分。}

\subsection*{13.$(1-\frac{y}{x})(x+y)^{2}$ 的展开式中$x^{2}y^{6}$ 的系数为(用数字作答)。}

\subsection*{14.写出与圆$x^{2}+y^{2}=1$ 和$(x-3)^{2}+(y-4)^{2}=16$ 都相切的一条直线的方程}



\subsection*{15.若曲线$y=(x+a)e^{x}$ 有两条过坐标原点的切线,则a的取值范围是}



\subsection*{16.已知椭圆$\begin{enumerate}
\item 
\end{enumerate}\frac{x^{2}}{a^{2}}+\frac{y^{2}}{b^{2}}=1(a>b>0)$ ,C的上顶点为A,两个焦点为$F_{1},F_{2}$ ,离$ 心$ 率为$\frac{1}{2}$ .过$F_{1}$ 且垂直于$A F_{2}$ 的直线与C交于D,E两点,$|DE|=6$ $\triangle ADE$ 的周长是}



\section*{四、 解答题:本题共6小题,共70分。解答应写出文字说明、证明过程或演算步骤。}17.(10分)

记$s_{n}$ 为数列$\{a_{n}\}$ 的前n项和,已知$a_{1}=1$ $\{\frac{S_n}{a_n}\}$ 是公差为$\frac{1}{3}$ 的等差数列.

\begin{enumerate}
\item 
\end{enumerate}求$\{a_{n}\}$ 的通项公式:

\begin{enumerate}
\item 
\end{enumerate}证明:$\frac{1}{a_{1}}+\frac{1}{a_{2}}+\cdots+\frac{1}{a_{n}}<2$ 

### 18. (12分)

记$\triangle ABC$ 的内角A,B,C的对边分别为a,$\pmb{b}$ ,c,已知$\frac{\cos A}{1+\sin A}=\frac{\sin2B}{1+\cos2B}$ 

\begin{enumerate}
\item 
\end{enumerate}若$c=\frac{2\pi}{3}$ ,求B;

\begin{enumerate}
\item 
\end{enumerate} 求$\frac{a^{2}+b^{2}}{c^{2}}$ 的最小值.

### 19. (12分)

如图,直三棱柱$ABC-A_{1}B_{1}C_{1}$ 的体积为4,$\triangle A_{1}BC$ 的面积为$2\sqrt{2}$ 



\begin{enumerate}
\item 
\end{enumerate}求A到平面$A_{1}BC$ 的距离:

\begin{enumerate}
\item 
\end{enumerate}设D为$A_{1}C$ 的中点,$AA_{1}=AB$ ,平面$A_{1}BC\perp$ 平面$ABB_{1}A_{1}$ ,求二面角$A-BD-C$ 的正弦值.



\includegraphics[width=0.6\textwidth]{imgs/img_in_image_box_811_1302_1138_1559.jpg}


---


2V.(一医疗团队为研究某地的一种地方性疾病与当地居民的卫生习惯(卫生习惯分为良好和不够良好两类)的关系,在己患该疾病的病例中随机调查了100例(称为病例组),同时在未患该疾病的人群中随机调查了100人(称为对照组),得到如下数据:


\begin{tabular}{c | c | c}
\hline
不够良好 & 良好 &  \\ 
\hline
病例组 & 40 & 60 \\ 
\hline
对照组 & 10 & 90 \\ 
\hline
\end{tabular}


\begin{enumerate}
\item 
\end{enumerate}能否有99%的把握认为患该疾病群体与未患该疾病群体的卫生习惯有差异?

\begin{enumerate}
\item 
\end{enumerate}从该地的人群中任选一人,A表示事件“选到的人卫生习惯不够良好”,B表示事件“选到的人患有该疾病”,$\frac{P(B\mid A)}{P(\overline{B}\mid A)} 与 \frac{P(B\mid\overline{A})}{P(\overline{B}\mid\overline{A})}$ 的比值是卫生习惯不够良好对患该疾病风险程度的一项度量指标,记该指标为R.



(i)证明:$R=\frac{P(A\mid B)}{P(\overline{A}\mid B)}\cdot\frac{P(\overline{A}\mid\overline{B})}{P(A\mid\overline{B})}$ 

(ii)利用该调查数据,给出$P(A\mid B)$ $P(A\mid\overline{B})$ 的估计值,并利用(i)的结果给出R的估计值.



$$K^{2}=\frac{n(ad-bc)^{2}}{(a+b)(c+d)(a+c)(b+d)},\quad\frac{P(K^{2}\geq k)\quad0.050\quad0.010\quad0.001}{k\quad3.841\quad6.635\quad10.828}$$

### 21. (12分)

已知点$A(2,1)$ 在双曲线\begin{enumerate}
\item 
\end{enumerate}$\frac{x^{2}}{a^{2}}-\frac{y^{2}}{a^{2}-1}=1(a>1).$ 上,直线I交C于P,Q两点,直线$\boldsymbol{A}\boldsymbol{P}$ ,AQ的斜率之和为0.



\begin{enumerate}
\item 
\end{enumerate}求1的斜率;

\begin{enumerate}
\item 
\end{enumerate}若$\tan\angle PAQ=2\sqrt{2}$ ,求$\triangle PAQ$ 的面积.

### 22. (12分)

已知函数$f(x)=\mathrm{e}^{x}-ax$ 和$g(x)=ax-\ln x$ 有相同的最小值.

\begin{enumerate}
\item 
\end{enumerate}求a;

(2)证明:存在直线$y=b$ ,其与两条曲线$y=f(x)$ 和$y=8(x)$ 共有三个不同的交点,并且从左到右的三个交点的横坐标成等差数列.



---


















\end{document}